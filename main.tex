\documentclass{beamer}
\usepackage[utf8]{inputenc}
\usepackage[T1]{fontenc}
\usetheme{metropolis}
\usecolortheme{beaver}


%\pagecolor{blue}
\title{Web 3.0}
\subtitle{The Future of the Internet\\(Made in LaTeX)}
\usepackage{tikz}
\titlegraphic { 
\begin{tikzpicture}[overlay,remember picture]
\node[left=0.3cm] at (current page.23){
    \includegraphics[width=2.3cm]{logo.png}
};
\end{tikzpicture}
}

\author{ by \\ \break\hspace{3mm}Raunak Kumar(2022CA077) \\ \&\hspace{1mm}Rishiraj Patel(2022CA081)\\}

\institute{\fontfamily{lmdh}\selectfont Motilal Nehru National Institute of Technology, Prayagraj}
\date{\today}

\begin{document}
\frame{\titlepage}

\begin{frame}{Introduction}
\setbeamercolor{block body}{bg=teal!20,fg=black}
\begin{block}{}
--> \hspace{3mm}Web 3.0 is the next generation of the internet, which promises to be more decentralized, secure, and user-centric.
\break\\
--> \hspace{3mm} It differs from previous versions of the web (Web 1.0 and Web 2.0) in that it is built on decentralized technologies like blockchain and aims to give users greater control over their data and digital assets.
\break\\
--> \hspace{3mm} Key features of Web 3.0 include decentralization, interoperability, privacy, and user ownership.

\end{block}
\end{frame}

\begin{frame}{Decentralization}
\setbeamercolor{block body}{bg=red!20,fg=black}
\begin{block}{}
--> \hspace{3mm}Decentralization is a key feature of Web 3.0, which means that there is no central authority controlling the network or its data.\break\\
--> \hspace{3mm}Blockchain technology enables decentralization by creating a trustless system where transactions are verified and recorded by a network of nodes.\break\\
--> \hspace{3mm}Decentralized applications (dApps) are built on blockchain technology and function without a central authority, allowing for greater transparency and security.
\end{block}
\end{frame}

\begin{frame}{What is Blockchain ?}
\setbeamercolor{block body}{bg=blue!20,fg=black}
\begin{block}{}
--> \hspace{3mm}Blockchain technology is the foundation of Web 3.0 and functions as a decentralized ledger that records transactions securely and transparently.\break\\
--> \hspace{3mm}Blockchain technology is immutable, meaning that once a transaction is recorded on the blockchain, it cannot be altered.\break\\
--> \hspace{3mm}Blockchain technology can be used to create decentralized applications (dApps) that are more secure, transparent, and efficient than traditional applications.
\end{block}
\end{frame}

\begin{frame}{Digital Currency a.k.a Crypto}
\setbeamercolor{block body}{bg=green!20,fg=black}
\begin{block}{}
--> \hspace{3mm}Cryptocurrencies are digital assets that are secured by cryptography and built on blockchain technology.\break\\
--> \hspace{3mm}Cryptocurrencies are a key feature of Web 3.0 because they enable decentralized transactions without the need for intermediaries like banks.\break\\
--> \hspace{3mm}Popular cryptocurrencies include Bitcoin, Ethereum, and Litecoin, and they are used for a variety of purposes, including investment, remittances, and payments.
\end{block}
\end{frame}

\begin{frame}{What are Smart Contracts ?}
\setbeamercolor{block body}{bg=purple!20,fg=black}
\begin{block}{}
--> \hspace{3mm}Smart contracts are self-executing contracts that are stored on a blockchain and automatically execute when certain conditions are met.\break\\
--> \hspace{3mm}Smart contracts enable automation and efficiency, reducing the need for intermediaries and allowing for greater transparency.\break\\
--> \hspace{3mm}Smart contracts can be used in a variety of industries, including finance, real estate, and supply chain management.
\end{block}
\end{frame}

\begin{frame}{Interoperability}
\setbeamercolor{block body}{bg=orange!20,fg=black}
\begin{block}{}
--> \hspace{3mm}Interoperability is the ability of different systems and networks to work together seamlessly.\break\\
--> \hspace{3mm}Interoperability is important in Web 3.0 because it enables different blockchain networks and dApps to communicate with each other.\break\\
--> \hspace{3mm}Interoperability can be achieved through standards and protocols, such as the Interledger Protocol and the Universal Wallet Interface.
\end{block}
\end{frame}

\begin{frame}{Privacy \& Security}
\setbeamercolor{block body}{bg=teal!20,fg=black}
\begin{block}{}
--> \hspace{3mm}Privacy is a key concern in Web 3.0 because traditional web technologies have failed to adequately protect user data.\break\\
--> \hspace{3mm}Decentralized technologies can help address privacy concerns by enabling users to own and control their data and digital assets.\break\\
--> \hspace{3mm}Technologies like zero-knowledge proofs and privacy-focused cryptocurrencies like Monero and Zcash can also help protect user privacy.
\end{block}
\end{frame}

\begin{frame}{Reliability}
\setbeamercolor{block body}{bg=yellow!20,fg=black}
\begin{block}{}
--> \hspace{3mm}User ownership is a key feature of Web 3.0 that aims to give users greater control over their data and digital assets.\break\\
--> \hspace{3mm}Decentralized technologies enable user ownership by creating trustless systems where users have full control over their data and digital assets.\break\\
--> \hspace{3mm}Examples of user ownership in Web 3.0 include decentralized identity solutions like uPort and SelfKey, and decentralized storage solutions like IPFS and Filecoin.
\end{block}
\end{frame}

\begin{frame}{Obstacles and Future Scope}
\setbeamercolor{block body}{bg=gray!20,fg=black}
\begin{block}{}
--> \hspace{3mm}Web 3.0 faces several challenges, including scalability, regulation, and adoption.\break\\
--> \hspace{3mm}However, Web 3.0 also presents several opportunities, including new business models, greater innovation, and a more decentralized and equitable internet.\break\\
--> \hspace{3mm}The potential impact of Web 3.0 is vast, with some experts predicting that it could transform the way we interact with technology and each other.
\end{block}
\end{frame}

\begin{frame}
\setbeamercolor{block body}{bg=orange!20,fg=black}{Conclusion}
\begin{block}{}
Web 3.0 is the future of the internet and promises to be more decentralized, secure, and user-centric than previous versions of the web. Blockchain technology, cryptocurrencies, smart contracts, interoperability, privacy, and user ownership are key features of Web 3.0. While there are challenges that need to be addressed, the potential benefits of Web 3.0 are significant, and it is likely to transform many industries in the coming years.
\end{block}
\end{frame}

\begin{frame}
\setbeamercolor{block body}{bg=green!20,fg=black}{Declaration}
\begin{block}{}
We are extremely glad to share this presentation and we hereby confirm that above information is true on our behalf.
We have taken help from various resources like Books,Youtube and reddit in order to complete this presentation with correct information.
Also, we want to state that this ppt is made in LaTeX in Overleaf editor.

Finally, We would like to Thank our Honorable Professor,Teachers and friends for guiding us .
\end{block}
\end{frame}

\end{document}





